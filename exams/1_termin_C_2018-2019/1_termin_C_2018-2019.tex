\documentclass[11pt,a4paper]{article}

\usepackage[left=2cm,text={17cm,24cm},top=2cm]{geometry}
\usepackage[english]{babel}
\usepackage[utf8]{inputenc}
\usepackage[T1]{fontenc}

\usepackage{amsmath}
\usepackage{amsfonts}

\pagenumbering{gobble}
\setlength\parindent{0pt}

\begin{document}

\begin{center}
  \begin{bf}
    \Huge{MAT}\\[-0.5em]
    {\normalsize\normalfont{1. termín 2018/2019}}\\[0.5em]

    \Large{skupina C}\\[-0.25em]
    {\normalsize\normalfont{(žltý papier)}}\\[1em]

    9. január 2019
  \end{bf}
\end{center}

\hrule

\subsection*{1 príklad (10b)}

  V jazyce teorie grupoidů s funkčním symbolem $f$ uveďte formuli, která je negací zákona o krácení převedenou do prenexního tvaru.\\

  \hrule

\subsection*{2 príklad (15b)}

  Buď \emph{L} jazyk s rovností, unárním funkčním symbolem \emph{f} a ternárním predikátovým symbolem \emph{p}. Uvažujme formule: $\varphi \equiv p(x,y,z) \rightarrow p(z,y,x)$, $\chi \equiv (x,y,x) \rightarrow x = y$, $\psi \equiv p(x, f(x), f(f(x)))$ a teorii $T = \lbrace \varphi, \chi, \psi \rbrace$.\\[2mm]
  (1) Nechť $\mathcal{M}$ je realizace jazyka L, jejímž univerzem je množina $\mathbb{R}$ všech reálných čísel, kde $f_{\mathcal{M}}(a) = a^2$ a $p_\mathcal{M}(a,b,c) \Leftrightarrow a \leq b \leq c$. Rozhodněte a) zda $\mathcal{M} \models \varphi$ , b) při jakém ohodnocení proměnných $e$ platí $\mathcal{M} \models \chi[e]$, c) zda $\mathcal{M} \models \neg \psi$.\\[2mm]
  (2) V realizaci $M$ navrhnete univerzum $\mathbb{R}$ nějakou jeho prodmožinu tak, aby vznikla realizace $\mathcal{M}'$, pro kterou bude platit $\mathcal{M}' \models T$.\\

  \hrule

\subsection*{3 príklad (15b)}

  Uvažujme grupoid $\mathcal{A} = (\mathbb{Z}, \ast)$, kde $x \ast y = \lfloor \frac{x+y}{2} \rfloor$ a $\lfloor x \rfloor$ značí dolní celou část reálného čísla $x$ (tj. největší celé číslo $y$ s vlastností $y \leq x$). Dále uvažujme grupoid $\mathcal{B} = (\mathbb{R}, \star)$ kde $x \star y = \frac{x+y}{2}$.\\[2mm]
  (1) Popište podgrupoid grupoidu $\mathcal{A}$ generovaný množinou $\{-2, 5\}$.\\
  (2) Určete nějaký podgrupoid grupoidu $\mathcal{B}$, který je homomorfním obrazem grupoidu $\mathcal{A}$.\\

  \hrule

\subsection*{4 príklad (15b)}

  Napište tabulku operace násobení v $GF(4)$. Jako ireducibilní polynom použijte $x^2 + x + 1$ a prvky tělesa $GF(4)$ vyjádřete v tabulce jako vektory se souřadnícemi v bázi $\{1, \alpha\}$, kde $\alpha$ je primitivní prvek.\\

  \hrule

\subsection*{5 príklad (10b)}

  Buď $\rho$ binární relace na množine $X$ a pro libovolné $x,y \in X$ položme

  \begin{center}
    $d(x,y) = \Bigg\{ \begin{tabular}{@{\hspace{-0.25mm}}l@{,\hspace{5mm}}c@{\hspace{1mm}}l}
                        $3$ & jestliže & $x \rho y$,\\
                        $0$ & jestliže & $\neg(x \rho y)$.
                      \end{tabular}
    $
  \end{center}

  Určete nutnou a postačující podmínku (vlastnosť relace $\rho$) pro to, aby $d$ byla metrika na množine $X$.\\

  \hrule

  \newpage

  \hrule

\subsection*{6 príklad (15b)}

  V souvislém rovinném grafu mají všechny uzly stejný stupeň, který je sudý, a počet buněk je 98. Určete počet hran tohoto grafu.\\

  \hrule

\end{document}
