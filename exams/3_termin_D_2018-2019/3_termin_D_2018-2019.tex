\documentclass[11pt,a4paper]{article}

\usepackage[left=2cm,text={17cm,24cm},top=2cm]{geometry}
\usepackage[english]{babel}
\usepackage[utf8]{inputenc}
\usepackage[T1]{fontenc}

\usepackage{amsfonts}
\usepackage{csquotes}

\pagenumbering{gobble}
\setlength\parindent{0pt}

\begin{document}

\begin{center}
  \begin{bf}
    \Huge{MAT}\\[-0.5em]
    {\normalsize\normalfont{3. termín 2018/2019}}\\[0.5em]

    \Large{skupina D}\\[-0.25em]
    {\normalsize\normalfont{(biely papier)}}\\[1em]

    28. január 2019
  \end{bf}
\end{center}

\hrule

\subsection*{1 príklad (15b)}

  Dokažte formuli

  \begin{center}
    $\vdash (A \rightarrow (B \rightarrow C)) \rightarrow (B \rightarrow (A \rightarrow C))$
  \end{center}

  Návod: Použijte vhodné předpoklady a opakovaně užijte odvozovací pravidlo a větu o dedukci.\\

  \hrule

\subsection*{2 príklad (10b)}

  Buď $L$ jazyk s predikátovými symboly $p$ a $q$ arity $2$ a $r$ arity $1$. Převeďte formuli

  \begin{center}
    $(\forall x p(x,y) \rightarrow \forall x \exists y q(x,y)) \rightarrow \forall x r(x)$
  \end{center}

  do prenexního tvaru.\\

  \hrule

\subsection*{3 príklad (15b)}

  Doplňte Caleyovu tabulku operace grupy na množině $\{a,b,c,d,e\}$. (Nápověda: Po nalezení neutrálneho prvku užijte opakovaně vlastnost tabulky, která plyne ze zákona o dělení (resp. krácení).)

  \begin{center}
    \begin{tabular}{c||c|c|c|c|c|}
          & $a$ & $b$ & $c$ & $d$ & $e$ \\
      \hline
      \hline
      $a$ &     &     &     & $b$ & $c$ \\
      \hline
      $b$ &     &     &     &     &     \\
      \hline
      $c$ & $d$ &     &     &     &     \\
      \hline
      $d$ &     &     &     &     &     \\
      \hline
      $e$ &     &     &     & $a$ &     \\
      \hline
    \end{tabular}
  \end{center}

  \hrule

\subsection*{4 príklad (15b)}

  Na algebře $\mathcal{A} = (\mathbb{Z},q,\ast)$ typu $(1,2)$ s operacemi definovanými pro $a,b \in \mathbb{Z}$ následovně

  \begin{center}
    $q(a) = 1 - a, \ a \ast b = a + b - ab$
  \end{center}

  uvažujme ekvivalenci $\rho$ danou vztahem $(a,b) \in \rho \Leftrightarrow 2 | (a - b)$, kde symbol $|$ znamená \enquote{delí}.\\[2mm]
  (a) Dokažte, že $\rho$ je kongruence na $\mathcal{A}$.\\
  (b) Popište faktorovou algebru $\mathcal{A} / \rho$, tj. určete její nosnou množinu (výčtem prvků) a všechny operace (jejich tabulkami).\\

  \hrule

  \newpage

  \hrule

\subsection*{5 príklad (15b)}

  Uvažujme na $\mathbb{R}^2$ normy $\Vert x \Vert_0$ a $\Vert x \Vert_1$ dané vztahy $\Vert x \Vert_0 = max \{|x_1|, |x_2|\}$ a $\Vert x \Vert_1 = |x_1| + |x_2|$ pro každé $x = (x_1, x_2) \in \mathbb{R}^2$. V kartézské soustavě souřadnic zakreslete množinu všech bodů $x \in \mathbb{R}^2$ s vlastností $\Vert x \Vert_0 \geq 2$ a $\Vert x \Vert_1 \leq 4$.\\

  \hrule

\subsection*{6 príklad (10b)}

  Kostra grafu $G$ má $21$ hran a každý jeho uzel má stupeň $5$. Určete počet hran grafu $G$.\\

  \hrule

\end{document}
