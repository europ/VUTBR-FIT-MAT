\documentclass[11pt,a4paper]{article}

\usepackage[left=2cm,text={17cm,24cm},top=2cm]{geometry}
\usepackage[english]{babel}
\usepackage[utf8]{inputenc}
\usepackage[T1]{fontenc}

\pagenumbering{gobble}

\begin{document}

\begin{center}
  \begin{bf}
    \Huge{MAT}\\[-0.5em]
    {\normalsize\normalfont{2. termín 2018/2019}}\\[0.5em]

    \Large{skupina C}\\[-0.25em]
    {\normalsize\normalfont{(modrý papier)}}\\[1em]

    21. január 2019
  \end{bf}
\end{center}

\hrule

\subsection*{1 príklad (15b)}

  Nech $L$ je jazyk s rovností a binárnym predikátovým symbolem $\subset$. Pri realizaci symbolu $\subset$ jako relace ostré inkluze na univerzu všech podmnožin nejaké množiny $M$ vyjadřete formulí jazyka L vlastnosť, že množiny reprezentované premennými $x$ a $y$ majú neprázdny prienik.\\

  \hrule

\subsection*{2 príklad (10b)}

  Dokážte, že pre uzavretú formulu $\varphi$ a formulu $\psi$ platí $\varphi \rightarrow \psi \vdash \varphi \rightarrow \forall x \psi$\\

  \hrule

\subsection*{3 príklad (15b)}

  Uvažujme pologrupu $A=(N_0^+,+)$ a $B=(N_0^+,\cdot)$. Nájdite homomorfizmus $f:A \rightarrow B$ kde $f(1)=3$. Zistite, zda je $f$ dokonca homomorfizmom algebry $(N_0^+, min, +)$ typu $(2,2)$ do algebry $(N_0^+, NSD, \cdot)$ typu $(2,2)$. Rozhodnite (s odvôvodnením), zda je nekterá s techto algeber okruhom.\\

  \hrule

\subsection*{4 príklad (15b)}

  Uveďte všetky triedy ľavého rozkladu a všetky triedy pravého rozkladu symetrickej grupy $S_3$ podľa jej podgrupy $\{(1),(1,3)\}$.\\

  \hrule

\subsection*{5 príklad (15b)}

  Buď $X$ unitárny prostor se skalárním součinem $(-,-)$ a $x,y \in X$. Určte $\Vert x \Vert$, jesliže $\rho(x,y)=9$, $(x,y)=-8$ a $\Vert y \Vert = 4$. Zde $\Vert \ \Vert$ zančí normu danou příslušným skalárním součinem a $\rho$ značí metriku danú normou $\Vert \ \Vert$.\\

  \hrule

\subsection*{6 príklad (10b)}

  Nakreslite graf s najmenším počtom uzlov ktorý nie je rovinný, ale je eulerovský a hamiltonovský.\\

  \hrule

\end{document}
